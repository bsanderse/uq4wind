
\documentclass[review]{elsarticle}
\usepackage{lineno,hyperref}
\modulolinenumbers[5]
\journal{ }
\usepackage{microtype}
\usepackage{algorithm}
\usepackage[noend]{algpseudocode}
\usepackage[a4paper, total={6.2in, 9in}]{geometry}
\usepackage[a4paper]{geometry}
\renewcommand{\baselinestretch}{1.2} 
\usepackage{color}
\usepackage[x11names]{xcolor}
\usepackage[utf8]{inputenc}
\usepackage{amsmath}
\usepackage{tabularx,ragged2e,booktabs,caption}
\renewcommand{\figurename}{{Figure}}
\usepackage[labelfont=bf]{caption}
\usepackage[font=footnotesize]{caption}
\usepackage{amssymb,latexsym}
\usepackage{graphicx}
\usepackage{booktabs}
\usepackage{amsfonts}
\usepackage{tikz}
\usepackage{array}
\usepackage{mathtools}
\usepackage{relsize}
\usepackage[T1]{fontenc}
\usepackage{cases}
\usepackage{amsthm}
\usepackage{subcaption}
\usepackage{multirow}
\usepackage[toc,page]{appendix}
\newcolumntype{Y}{>{\centering\arraybackslash}X}
\newtheorem{theorem}{Theorem}
\newtheorem{remark}{Remark}[section]
\numberwithin{equation}{section}
\usepackage{tabularx,ragged2e,booktabs,caption}
\newcommand{\footremember}[2]{%
    \footnote{#2}
    \newcounter{#1}
    \setcounter{#1}{\value{footnote}}%
}
\newcommand{\footrecall}[1]{%
    \footnotemark[\value{#1}]%
}
\usepackage[toc]{appendix}
\DeclareMathAlphabet{\mathpzc}{OT1}{pzc}{m}{it}
\newcommand{\lnorm}{\left|\left|}
\newcommand{\rnorm}{\right|\right|}
\newcommand{\Ld}{L^2(\mathcal{D})}
\newcommand{\Lomtd}{L^2(\Omega;\mathcal{T};\mathcal{D})}
%\newcommand{\Lomd}{L^2(\Omega;L^2(\mathcal{D}))}
\newcommand{\Lomd}{L^2(\Omega,\mathcal{D})}
\newcommand{\myreferences}{../Bibliography/references}
\bibliographystyle{elsarticle-num}
\numberwithin{equation}{section}
\usepackage{tabularx,ragged2e,booktabs,caption}
\graphicspath{{./images/}}
\usepackage[toc]{appendix}
\DeclareMathAlphabet{\mathpzc}{OT1}{pzc}{m}{it}
\bibliographystyle{elsarticle-num}
\usepackage{stix2}
\usepackage{pgfplots}
\usepackage{appendix}
\usepackage[labelfont=bf]{caption}
\pgfplotsset{compat=newest}
\pagestyle{myheadings}
%%%%%%%%%%%%%%%%%%%%%%%
\title{A Global Sensitivity Analysis of Wind Turbine Aeroelastic Models}
\begin{document}
\begin{frontmatter}
%\address[cwi]{CWI - Centrum Wiskunde \& Informatica, Amsterdam, The Netherlands}
%\address[aero]{Faculty of Aerospace Engineering, Delft University of Technology, Delft, The Netherlands.}
%\author[cwi,aero]{Prashant~Kumar}
\author[]{Prashant~Kumar}
\ead{pkumar@cwi.nl}
\begin{abstract}
Wind turbines are a complex multi-physics systems. As the performance and the life span of a turbine is highly dependent on  manufacturing, meteorological and operational factors, it is very important to quantify the impact of each of these parameter in the output of interest.  
\end{abstract}
\begin{keyword}
Global sensitivity analysis, UQ, Sobol indices, aeroelastic models
\end{keyword}
\end{frontmatter}
\linenumbers
\section{Introduction}
Aeroelastic models such as the Blade Element Momentum (BEM) models \cite{HandBook} play a critical role in the design, development and optimization of modern wind turbines. In particular, BEM models are employed to predict turbine response such as the structural loads and power outputs. A number of parameters describing meteorological and operational conditions as well as manufacturing specifications are needed as input to simulate BEM models. In this work, we seek to quantify the sensitivities of these input parameters for different turbine responses.
\section{Aeroelastic models}
\subsection{BEM models}
\subsection{Input parameters of aeroelastic models}
\section{Global sensitivity analysis}
The main objective of sensitivity analysis is to quantify the relative significance of individual inputs (or in combination) and how variations in input values affects the output of interest. In engineering, sensitivity analysis can be employed for a number of reasons: to determine the stability and robustness of a computational model with respect to input parameters, for simplification of stochastic models by fixing the insensitive parameters, and to guide data acquisition campaigns and experimental design to refine the data on sensitive parameters.

Sensitivity analysis techniques can be classified as local and global methods, see \cite{ralphSmith}. In local sensitivity analysis, individual parameters are perturbed around their nominal values --- allowing for the description of output variability only in a small neighbourhood of nominal input values. Although, local approaches are widely employed due to their ease of implementation and low computational cost, they are unable to quantify global behavior of nonlinearly parametrized models such as aeroelastic models. Global sensitivity approaches, on the other hand, consider the entire range of input values to compute output sensitivities. Therefore, global sensitivity analysis is more suitable for aeroelastic models considered in this work.  

\subsection{Sobol analysis}
In this work, we employ variance-based Sobol decomposition to perform global sensitivity analysis. One of the advantages of this method is that the relative importance of input parameters is quantified on a scale of $[0,1]$ known as \emph{Sobol indices}. The main idea of Sobol analysis is to decompose the total variance of the output in term of contributions from individual parameters and their contribution. For the ease of exposition, let us consider a nonlinear model:
\begin{equation}\label{nonlinear_model}
Y = f(\mathbf{Z}),
\end{equation}
where $\mathbf{Z} = [z_1, z_2, ..., z_M]\in \mathcal{D}_{\mathbf{Z}}\in \mathbb{R}^M$ is the input vector. For simplicity, we assume input parameters are uniformly distributed i.e. $z_i \sim \mathcal{U}(0,1)$ and the support of input set is $\mathcal{D}_\mathbf{Z}  =  [0,1]^M$ where $M$ is the total number input parameters. The Sobol decomposition is defined as:
\begin{equation}\label{sobol_decomp}
f(z_1, z_2, ..., z_M) = f_0+\sum_{i=1}^M f_i(z_i) + \sum_{1\leq<i<j\leq M} f_{ij}(z_i,z_j) + ... + f_{1, 2, ..., M}(z_1, z_2, ..., z_M),
\end{equation}
The above decomposition is only valid for independent input parameters. 
\subsubsection{PCE Ordinary Least Square (PCE\_OLS)}
\subsubsection{PCE Least Angle Regression (PCE\_LAR)}
\section{Parametrization of uncertain inputs in the BEM model}
\subsection{Geometric uncertainty}
We use Non-Uniform Rational Basis Splines (NURBS) \cite{nurbs_book} to perturb the geometrical parameters of the turbine blade, such as the reference chord and twist curves. The main advantage of using NURBS is that it provides great flexibility to approximate a large variety of curves with a limited number of control points. Further, the set of control points and knots can be directly manipulated to control the smoothness and curvature.

\subsubsection{NURBS based perturbation}
The value of NURBS curve at each location $x$ is computed using a weighted sum of $N$ basis functions (or B-splines):
\begin{equation}\label{NURB_curve}
S(x)  = \sum_{i=1}^{N} c_i B_{i,p}(x),
\end{equation}
where $S(x)$ is the value of the curve at location $x, c_i$ is the weight of control point $i$ and $B_{i,p}(x)$ is the value of B-spline corresponding to the $i$-th control point at $x$. The subscript $p$ denote the polynomial degree of the NURBS curve. The total number of control points $N = m+p+1$ where $m$ is the number of knots  and the degree of NURBS curve. The above definition can be easily extended to higher dimensions. 

The B-splines are recursive in polynomial degree, for example, we can derive quadratic B-splines ($p=2$) using linear B-splines ($p=1$), cubic ($p=3$) from quadratic B-splines and so on. Given $m$ knot locations $t_1, t_2, ..., t_{m}$, the B-spline of degree 0 is defined as:
\begin{equation}\label{linearBspline}
B_{i,0}(x) :=
\begin{cases}
1\quad t_i\leq x < t_{i+1},\quad i = 1,2,...,m,\\
0\quad\text{elsewhere.}
\end{cases} 
\end{equation}
Higher order B-splines can then be derived using the recurrence relation \cite{deBoor}:
\begin{equation}\label{NURBS_recurrence}
B_{i,p}(x) := \frac{x-t_i}{t_{i+p} - t_i}B_{i,p-1}(x) + \frac{t_{i+p+1}  -  x}{t_{i+p+1}  -  t_{i+1}}B_{i+1,p-1}(x),\quad p\geq1, i = 1,2,...,m.
\end{equation} 
Something about padding...
These B-splines can be constructed efficiently using the De Boor's algorithm \cite{deBoor}. In Fig. \ref{B-splines}, we show linear, quadratic and cubic splines for $x\in[0,1]$. At the interval boundaries these B-splines go to zero smoothly. The domain of influence of a given control point depends on the respective B-spline (rephrase)??

Next, we describe steps to generate perturbed samples of chord from a given reference chord, $S_{ref}(x)$, using NURBS based parametrization. The first step is to approximate the given reference chord using a NURBS curve with a fixed degree $p$. For this, we need to sample $S_{ref}(x)$ at $N$ locations $\{x_j\}_{j=1}^N$ and compute $B_{i,p}(x_j), \text{ for }i, j = 1,2, ..., N$, such that we can compute the set of control points $\mathbf{c}=\{c_i\}_{i=0}^N$ by solving the following linear system:
\begin{equation}\label{nurbs_inversion}
\mathbf{B}\mathbf{c} = \mathbf{S},
\end{equation}
where $\mathbf{S}\in \mathbb{R}^{N}$ is a vector containing sampled values of the reference curve and $\mathbf{B}\in \mathbb{R}^{N\times N}$ is a matrix with $j$-th row consisting of B-splines values at $x_j$ locations, i.e., $B_{i,p}(x_j), i= 1, 2,..., N$. 
Once the control points are obtained, we can derive the approximate reference curve $S_N(x)$ using \eqref{NURB_curve}. Note that the accuracy of the approximated curve is dependent on the sample locations $x_j$ as well as the degree $p$ and can be set heuristically \cite{adaptiveNurbs}. The number of sampled locations $N$ can be adaptively increased until the following tolerance criteria is met:
\begin{equation}
\frac{||S_N - S_{ref}||}{||S_{ref}||} <\varepsilon.
\end{equation}
More advanced approaches, monotonicity preserving methods, etc??
%\begin{remark}
%How to chose $x$?
%\end{remark}
%The final step is to introduce uncertainty in the above computed control points by associating a probability distribution.
%
%The steps to obtained perturbed samples from a given reference curve is outlined in Fig. \ref{}.
\subsection{Model uncertainty}
\subsection{Wind velocity}
\section{Description of test case}
\section{Sensitivity analysis workflow}
\section{Numerical experiment}
\subsection{Interpretation of global sensitivity analysis result}
\section{Conclusions}
\section*{Acknowledgements}
\end{document}