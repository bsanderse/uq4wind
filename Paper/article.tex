
\documentclass[review]{elsarticle}
\usepackage{lineno,hyperref}
\modulolinenumbers[5]
\journal{ }
\usepackage{microtype}
\usepackage{algorithm}
\usepackage[noend]{algpseudocode}
\usepackage[a4paper, total={6.2in, 9in}]{geometry}
\usepackage[a4paper]{geometry}
\renewcommand{\baselinestretch}{1.2} 
\usepackage{color}
\usepackage[x11names]{xcolor}
\usepackage[utf8]{inputenc}
\usepackage{amsmath}
\usepackage{tabularx,ragged2e,booktabs,caption}
\renewcommand{\figurename}{{Figure}}
\usepackage[labelfont=bf]{caption}
\usepackage[font=footnotesize]{caption}
\usepackage{amssymb,latexsym}
\usepackage{graphicx}
\usepackage{booktabs}
\usepackage{amsfonts}
\usepackage{tikz}
\usepackage{array}
\usepackage{mathtools}
\usepackage{relsize}
\usepackage[T1]{fontenc}
\usepackage{cases}
\usepackage{amsthm}
\usepackage{subcaption}
\usepackage{multirow}
\usepackage[toc,page]{appendix}
\newcolumntype{Y}{>{\centering\arraybackslash}X}
\newtheorem{theorem}{Theorem}
\newtheorem{remark}{Remark}[section]
\numberwithin{equation}{section}
\usepackage{tabularx,ragged2e,booktabs,caption}
\newcommand{\footremember}[2]{%
    \footnote{#2}
    \newcounter{#1}
    \setcounter{#1}{\value{footnote}}%
}
\newcommand{\footrecall}[1]{%
    \footnotemark[\value{#1}]%
}
\usepackage[toc]{appendix}
\DeclareMathAlphabet{\mathpzc}{OT1}{pzc}{m}{it}
\newcommand{\lnorm}{\left|\left|}
\newcommand{\rnorm}{\right|\right|}
\newcommand{\Ld}{L^2(\mathcal{D})}
\newcommand{\Lomtd}{L^2(\Omega;\mathcal{T};\mathcal{D})}
%\newcommand{\Lomd}{L^2(\Omega;L^2(\mathcal{D}))}
\newcommand{\Lomd}{L^2(\Omega,\mathcal{D})}
\newcommand{\myreferences}{../Bibliography/references}
\bibliographystyle{elsarticle-num}
\numberwithin{equation}{section}
\usepackage{tabularx,ragged2e,booktabs,caption}
\graphicspath{{./images/}}
\usepackage[toc]{appendix}
\DeclareMathAlphabet{\mathpzc}{OT1}{pzc}{m}{it}
\bibliographystyle{elsarticle-num}
\usepackage{stix2}
\usepackage{pgfplots}
\usepackage{appendix}
\usepackage[labelfont=bf]{caption}
\pgfplotsset{compat=newest}
\pagestyle{myheadings}
%%%%%%%%%%%%%%%%%%%%%%%
\title{A Global Sensitivity Analysis of Wind Turbine Aeroelastic Models}
\begin{document}
\begin{frontmatter}
%\address[cwi]{CWI - Centrum Wiskunde \& Informatica, Amsterdam, The Netherlands}
%\address[aero]{Faculty of Aerospace Engineering, Delft University of Technology, Delft, The Netherlands.}
%\author[cwi,aero]{Prashant~Kumar}
\author[]{Prashant~Kumar}
\ead{pkumar@cwi.nl}
\begin{abstract}
Wind turbines are a complex multi-physics systems. The performance and life-cycle of turbine is influenced by a large number
\end{abstract}
\begin{keyword}
Wind energy, global sensitivity analysis, UQ, aeroelastic models 
\end{keyword}
\end{frontmatter}
\linenumbers
\section{Introduction}
\section{Aeroelastic models}
\subsection{BEM models}
\subsection{Input parameters of aeroelastic models}
\section{Global sensitivity analysis}
\subsection{Sobol' indices}
\subsubsection{PCE Ordinary Least Square (PCE\_OLS)}
\subsubsection{PCE Least Angle Regression (PCE\_LAR)}
\section{Parametrization of uncertain inputs in the BEM model}
\subsection{Geometric uncertainty}
We use Non-Uniform Rational Basis Splines (NURBS) \cite{nurbs_book} to perturb the geometrical parameters of the turbine blade, such as the reference chord and twist curves. The main advantage of using NURBS is that it provides great flexibility to approximate a large variety of curves using a very limited number of control points. Further, the set of control points and knots can be directly manipulated to control the smoothness and curvature.

\subsubsection{NURBS based perturbation}
The value of NURBS curve at each location $x$ is computed using a weighted sum of $N$ basis functions (or B-splines):
\begin{equation}\label{NURB_curve}
S(x)  = \sum_{i=0}^{N-1} c_iB_{i,p}(x),
\end{equation}
where $S(x)$ is the value of the curve at location $x$, $c_i$ is the weight of control point $i$ and $B_{i,p}(x)$ is the value of B-spline corresponding to  the $i$-th control point at $x$. The subscript $p$ denote the polynomial degree of the NURBS curve. The total number of control points $N= m+p+1$ where $m$ is the number of knots  and the degree of NURBS curve. The above definition can be easily extended to higher dimensions. 

The B-splines are recursive in polynomial degree, for example, we can derive quadratic B-spline ($p=2$) using the linear B-spline ($p=1$); cubic ($p=3$) from quadratic B-splines and so on. Given $m$ knot locations $t_0,t_1,...,t_{m-1}$, the B-spline of degree 0 is defined as:
\begin{equation}\label{linearBspline}
B_{i,0}(x) :=
\begin{cases}
1\quad t_i\leq x < t_{i+1},\\
0\quad\text{elsewhere.}
\end{cases} 
\end{equation}
Higher order B-splines can then be derived using the recurrence relation \cite{deBoor}:
\begin{equation}\label{NURBS_recurrence}
B_{i,p}(x) := \frac{x-t_i}{t_{i+p} - t_i}B_{i,p-1}(x) + \frac{t_{i+p+1}  -  x}{t_{i+p+1}  -  t_{i+1}}B_{i+1,p-1}(x).
\end{equation}
These B-splines can be constructed efficiently using the De Boor's algorithm \cite{deBoor}. In Fig. \ref{B-splines}, we show the linear, quadratic and cubic spline for $x\in[0,1]$. At the interval boundaries these B-splines go smoothly to zero. 

Next, we describe steps to generate perturbed samples of chord from a given reference chord, $S_{ref}(x)$, using NURBS based parametrization. The first step is to approximate the given reference chord using a NURBS curve with a fixed degree $p$. For this, we sample $S_{ref}(x)$ at $N$ locations $\{x_i\}_{i=0}^N$ and


 perform inversion to compute $\{c_i\}_{i=0}^N$ by solving the $N\times N$ linear system:
\begin{equation}\label{nurbs_inversion}
\mathbf{B}\mathbf{c} = \mathbf{S},
\end{equation}
where $\mathbf{B}\in \mathbb{R}^{N\times N}$ with elements $\mathbf{B}_{ij} = B_{}()$

with a \emph{limited} number of control points. The number of control points are decided based on an error threshold criteria between the approximated and the reference chord. 
\begin{remark}
How to chose $x$
\end{remark}

\subsection{Model uncertainty}
\subsection{Wind velocity}
\section{Description of test case}
\section{Sensitivity analysis workflow}
\section{Numerical experiment}
\section{Interpretation of global sensitivity analysis result}
\section{Conclusions}
\section*{Acknowledgements}
\end{document}