\section{Literature overview of global sensitivity analysis in wind-turbine models}
List of BEM codes in \cite{Sayed2019,Vorpahl2013}.

Uncertainties and corrections in BEM models \cite{Madsen2012}: complex inflow, e.g. sheared inflow.

\cite{Eggers2003} study the effect of shear and turbulence on rotor loading, but not with a systematic (global) sensitivity analysis.

\cite{McKay2014} perform a measurement data-only global sensitivity analysis using a neural network and a Fourier amplitude sensitivity test (a similar technique was used in \cite{Kusiak2010} for vibration analysis). The neural network is constructed based on measurement data. The output quantity of interest is the power production; several input parameters are used, including yaw angle, rotor speed, blade pitch angle, wind speed, ambient temperature, main bearing temperature, wind speed standard deviation and yaw angle standard deviation.

\cite{Dykes2014} performed global sensitivity analysis of turbine costs with respect to key wind turbine configuration parameters including rotor diameter, rated power, hub height, and maximum tip speed. DAKOTA is used to calculate Sobol indices with Monte Carlo sampling.

\cite{Rinker2016a} calculated the global sensitivity of wind turbine loads to wind inputs using polynomial response surfaces. She used Sobol indices, with quantity of interest the maximum blade root bending moment and as inputs four turbulence parameters: a reference mean wind speed, a reference turbulence intensity, the Kaimal length scale, and a parameter reflecting the nonstationarity. The turbine model studied is the WindPACT 5 MW reference model and the BEM model is FAST.

\cite{Echeverria2017} employs a global sensitivity analysis to identify the design variables that affect wind turbine performance in order to reduce the number of variables in wind turbine design optimization. As input parameters, airfoil parameterization and chord and twist distributions are used; the outputs studied are annual energy production, maximum blade tip deflection, overall sound power level and blade mass. For example, Sobol indices are shown for the effect of chord and twist control points on the output quantities.

\cite{Matthaus2017} performed uncertainty quantification including uncertainty in the aerodynamic properties; for example, the uncertainty in the lift coefficient at an airfoil section due to uncertain roughness (e.g.\ contamination) is achieved by interpolating between clean and fully rough state with a uniformly distributed random variable. Polynomial chaos and Kriging were used. See also \cite{Bortolotti2019}, in which the AVATAR turbine was analysed, using the Cp-Lambda aeroelastic model and DAKOTA as UQ toolbox. 

\cite{Murcia2018} similarly considered a global sensitivity analysis with Sobol indices computed by using sparse polynomial chaos expansion. The quantities of interest considered are the energy production and lifetime equivalent fatigue loads for the DTU 10 MW reference turbine. The uncertain input is given by the turbulent inflow field, with 4 parameters: mean hub height wind speed, std. dev. of hub height wind speed, shear exponent, yaw misalignment. The dependency between these parameters (as described by the Normal Turbulence Model) is taken into account by using a Rosenblatt transformation that transforms the dependent variables to a set of independent ones. The Chaospy package is used for the analysis. Seven different model outputs are considered: power, thrust, and several damage equivalent fatigue loads (EFL), computed using a rainflow counting algorithm. The sensitivity analysis shows that the turbulent inflow realization has a bigger impact on the total distribution of equivalent fatigue loads than the shear coefficient or yaw misalignment.

\cite{Sayed2019} argue that BEM models are accurate mainly for small turbines but that for large turbines, when the tip deformations exceed 10\% of the blade radius, it is questionable if they can still be applied. They compare the results of engineering models to CFD-based aeroelastic simulations. It is shown that the 2D polars commonly used in BEM require 3D corrections, and that including more polars improves the results. `Turbine power and thrust predicted from the flexible rotor were reduced compared to the rigid rotor employing a BEM-based model. A reason behind the increase in power from the CFD-based model is the spanwise force component. It is not included in the BEM-based simulations as it is one of the main assumptions of its theory. This assumption could be used for small wind turbine simulations where the edgewise deformations were insignificant. But for such large wind turbines, the spanwise force distribution over the blade radius must be taken into account in the calculation of the power output due to the large edgewise deformations. To increase the accuracy of the engineering model, it is recommended to increase the number of polars calculated by 2D CFD simulations at different sections along the blade. This is because of the significant difference in the Reynolds number for the large blade.'

\cite{Velarde2019} perform a global sensitivity analysis of fatigue loads with respect to structural, geotechnical and metocean parameters for a 5MW offshore wind turbine installed on a gravity based foundation. Linear regression of Monte Carlo simulations and Morris screening are performed for three design load cases. 

\cite{Robertson2018} assesses the sensitivity of different wind parameters on the loads of a wind turbine. They argue that the most common parameters considered normally in literature are the turbulence intensity variability, and the shear exponent, or wind profile, both having important influence on the turbine response. In the paper an assessment of which wind characteristics influence wind turbine structural loads is given. The correlated dependency between the individual parameters is not considered; rather, they focus on assessing the sensitivity of each parameter individually. An elementary-effects screening method (Morris screening) is used as global sensitivity method. The NREL 5MW turbine and FAST are used.

\cite{Fluck2018} develop an \textit{intrusive} stochastic BEM method: `a stochastic solution for the unsteady aerodynamic loads based on a projection of the unsteady Blade Element Momentum (BEM) equations onto a stochastic space spanned by chaos exponentials'.

\cite{Hubler2017} used global sensitivity analysis (as part of a four-step approach) to reduce the parameter dimension for structural behaviour computations using FAST as aeroelastic code. The methods of expert knowledge, one-at-a-time variation, and regression based on Monte Carlo sampling are used as pre-processing steps to reduce the computational expense of global sensitivity analysis.